% LaTeX resume using res.cls
\documentclass[line,margin]{res}
\usepackage[T1]{fontenc}
\usepackage{charter}
\usepackage{inconsolata}
\usepackage{url}
\usepackage[official]{eurosym}
\usepackage{booktabs}
\usepackage{fancyhdr}

\pagestyle{fancy}
\renewcommand{\headrulewidth}{0pt}
\fancyhf{}
\rfoot{\rule{14.7em}{0.4pt} \\ Updated December 2024~|~\thepage~of~\pageref{LastPage}}

\setlength{\parskip}{0.6em}

\begin{document}

\name{\Huge{Trevor Bekolay} \hspace{7.5em} \normalsize{tbekolay@gmail.com}}

\begin{resume}

\section{CURRENT \\ POSITION}

{\sl Head of Application Engineering,} Applied Brain Research \\
December 2023--present
\begin{itemize} \itemsep -2pt
  \item Work with customers to define requirements for integrating
    the TSP chip hardware in their consumer devices.
  \item Develop software to support customer integrations.
\end{itemize} \vspace{-4pt}

{\sl Senior research scientist,} Applied Brain Research \\
February 2016--December 2023
\begin{itemize} \itemsep -2pt
  \item Led development of Nengo software, which includes
    writing code, managing contributions,
    releasing packages, and communicating with the public.
  \item Research learning algorithms,
    speech recognition, and speech synthesis
    in spiking neural networks.
\end{itemize} \vspace{-4pt}

\section{EDUCATION}

{\sl Doctor of Philosophy,} Computer Science \\
University of Waterloo, Waterloo, ON, \\
graduated February 2016 \\
with Graduate Diplomas in Theoretical Neuroscience and Cognitive Science \\
Member of the Computational Neuroscience Research Group \\
Cumulative GPA: 94.00

{\sl Master of Mathematics,} Computer Science \\
University of Waterloo, Waterloo, ON, \\
graduated October 2011 \\
Member of the Computational Neuroscience Research Group \\
Cumulative GPA: 91.50

{\sl Honours Bachelor of Computer Science, Co-op Option} \\
University of Manitoba, Winnipeg, MB, \\
graduated May 2009 \\
Minor in Asian Studies \\
Cumulative GPA: 4.28/4.5

\section{AWARDS \& \\ SCHOLARSHIPS}

\hspace{-1.4em}\begin{tabular}{lrl}
{\sl David R. Cheriton Grad. Scholarship} (CS, UWaterloo) &
\$20,000 & 2015/01 -- 2015/12\\
{\sl Travel Award for Cognitive Science 2013} (AI Journal) &
\euro{}500 & 2013/07 \\
{\sl David R. Cheriton Grad. Scholarship} (CS, UWaterloo) &
\$20,000 & 2013/01 -- 2014/12\\
{\sl Alexander Graham Bell CGS - Doctorate} (NSERC) &
\$105,000 & 2012/09 -- 2015/09 \\
{\sl President's Graduate Scholarship} (CS, UWaterloo) &
\$30,000 & 2012/09 -- 2015/09 \\
{\sl Ontario Graduate Scholarship} (Gov. of Ontario) &
Declined & 2012/09 -- 2013/09 \\
{\sl Ontario Graduate Scholarship} (Gov. of Ontario) &
\$15,000 & 2011/09 -- 2012/09 \\
{\sl President's Graduate Scholarship} (CS, UWaterloo) &
\$10,000 & 2011/09 -- 2012/09 \\
{\sl OGS in Science and Technology} (Gov. of Ontario) &
\$5,000 & 2011/01 -- 2011/05 \\
{\sl Alexander Graham Bell CGS - Masters} (NSERC) &
\$17,500 & 2009/09 -- 2010/09 \\
{\sl President's Graduate Scholarship} (CS, UWaterloo) &
\$10,000 & 2009/09 -- 2010/09 \\
{\sl Distinguished TA Award} (CS, UWaterloo) &
-- & 2010/09 \\
{\sl Excellence in Teaching Assistance Award} (UManitoba) &
\$300 & 2008/09 \\
{\sl Shell FUEL Scholarship} (UManitoba) &
\$3,750 & 2008/09 -- 2009/04 \\
{\sl Undergraduate Student Research Award} (NSERC) &
\$4,500 & 2008/05 -- 2008/08 \\
{\sl Elizabeth Luginbuhl Memorial Award} (UManitoba) &
\$2,250 & 2007/09 -- 2008/09 \\
{\sl Undergraduate Student Research Award} (NSERC) &
\$4,500 & 2007/05 -- 2007/08 \\
{\sl Japan Student Services Scholarship} (Kokugakuin) &
\$11,000 & 2005/08 -- 2006/08
\end{tabular}

\section{TEACHING \\EXPERIENCE}

{\bf Sessional Instructor} ~~3 terms \vspace{4pt} \\
David R. Cheriton School of Computer Science, University of Waterloo
\begin{itemize}  \itemsep -2pt
  \item CS230 -- Introduction to Computers and Computer Systems \hfill Fall 2011
  \item CS116 -- Introduction to Computer Science 2 \hfill Winter 2011
\end{itemize} \vspace{-4pt}
Department of Computer Science, University of Manitoba
\begin{itemize}  \itemsep -2pt
  \item COMP1010 -- Introductory Computer Science 1 \hfill Fall 2008
\end{itemize}

\clearpage

{\bf Workshop Instructor} ~~3 two day workshops \vspace{4pt} \\
Software Carpentry, Mozilla Foundation

\begin{itemize} \itemsep -2pt
  \item CAPS Computational Biology Laboratory. Columbus, OH. \hfill
    October, 2014
  \item General audience at PyCon. Montreal, QC. \hfill April, 2014
  \item General audience at Mozilla. Toronto, ON. \hfill February, 2014
\end{itemize}

{\bf Tutorial/Lab Instructor} ~~9 terms \vspace{4pt} \\
David R. Cheriton School of Computer Science, University of Waterloo
\begin{itemize}  \itemsep -2pt
  \item CS230 -- Intro to Comp. \& Comp. Systems \hfill Fall 2012, Fall 2011, Winter 2010
  \item CS115 -- Introduction to Computer Science 1 \hfill Fall 2010
\end{itemize} \vspace{-4pt}
Department of Computer Science, University of Manitoba
\begin{itemize}  \itemsep -2pt
  \item COMP2160 -- Programming Practices \hfill Fall 2008, Fall 2007
  \item COMP2140 -- Data Structures and Algorithms \hfill Winter 2008
  \item COMP1010 -- Introductory Computer Science 1 \hfill Fall 2007
  \item COMP1260 -- Introductory Computer Usage 1 \hfill Fall 2006
\end{itemize}

{\bf Marker} ~~8 terms \vspace{4pt} \\
David R. Cheriton School of Computer Science, University of Waterloo
\begin{itemize}  \itemsep -2pt
  \item CS486/686 -- Introduction to Artificial Intelligence \hfill Summer 2010
  \item CS135 -- Designing Functional Programs \hfill Fall 2009
\end{itemize} \vspace{-4pt}
Department of Computer Science, University of Manitoba
\begin{itemize}  \itemsep -2pt
  \item COMP3040 -- Technical Communication in CS \hfill Winter 2009, Winter 2008
  \item COMP3620 -- Professional Practice in Computer Science \hfill Winter 2009
  \item COMP2140 -- Data Structures and Algorithms \hfill Winter 2008, Winter 2007
  \item COMP1010 -- Introductory Computer Science 1 \hfill Fall 2006
\end{itemize}

\section{TEACHING \\ACCREDITATION}

{\bf Certificate in University Teaching} \\
Completed in August, 2015.

{\bf Fundamentals of University Teaching} \\
Completed in April, 2012.

\section{WORK \\EXPERIENCE}

{\sl Staff Writer} \hfill Winter 2010 -- Summer 2011 \\
How-To Geek, LLC, Herndon, VA

{\sl Science and Technology Editor} \hfill 2008 -- 2009 \\
The Manitoban student newspaper, University of Manitoba

{\sl Research Assistant} \hfill Winter 2008 \\
Centres for Research in Youth, Science Teaching and Learning (CRYSTAL)

{\sl Research Assistant} \hfill Winter 2007 \\
Intelligent Reasoning, Critiquing, and Learning lab at the University of Alberta

\section{OUTREACH}

Guest on the Weekly Weinersmith podcast, December 9, 2012. \\
\url{http://www.weeklyweinersmith.com/?p=510}

{\Large \bf Refereed Publications} \\ \vspace{-8pt} \hrule

\section{JOURNAL \\ARTICLES}

V. Senft, T.C. Stewart, \textbf{T. Bekolay}, C. Eliasmith,
B. J. Kr\"{o}ger. (2018) \\
  {\sl Inhibiting basal ganglia regions reduces syllable sequencing errors
    in parkinson's disease: a computer simulation study.} \\
  Frontiers in Computational Neuroscience, 12:41.
  DOI: 10.3389/fncom.2018.00041

B. J. Kr\"{o}ger, E. Crawford, \textbf{T. Bekolay}, C. Eliasmith. (2016) \\
  {\sl Modeling interactions between speech production and perception:
  speech error detection at semantic and phonological levels
  and the inner speech loop.} \\
  Frontiers in Computational Neuroscience, 10:51.
  DOI: 10.3389/fncom.2016.00051

V. Senft, T.C. Stewart, \textbf{T. Bekolay}, C. Eliasmith,
B. J. Kr\"{o}ger. (2015) \\
  {\sl Reduction of dopamine in basal ganglia and its effects on
    syllable sequencing in speech: a computer simulation study.} \\
  Basal Ganglia, 6(1):7--17. DOI: 10.1016/j.baga.2015.10.003

\textbf{T. Bekolay}, M. Laubach, C. Eliasmith. (2014) \\
  {\sl A spiking neural integrator model of the adaptive control of action
  by the medial prefrontal cortex.}
  The Journal of Neuroscience, 34(5):1892--1902.

\textbf{T. Bekolay}, J. Bergstra, E. Hunsberger, T. DeWolf, T.C. Stewart,
  D. Rasmussen, X. Choo, A. R. Voelker, C. Eliasmith. (2014) \\
  {\sl Nengo: a Python tool for building large-scale functional
  brain models.} \\ Frontiers in Neuroinformatics, 7:48.
  DOI: 10.3389/fninf.2013.00048

C. Eliasmith, T.C. Stewart, X. Choo, \textbf{T. Bekolay},
  T. DeWolf, C. Tang, D. Rasmussen. (2012)
  {\sl A large-scale model of the functioning brain.} \\
  Science 338:6111, 1202--1205. DOI: 10.1126/science.1225266

T.C. Stewart, \textbf{T. Bekolay}, C. Eliasmith. (2012) \\
  {\sl Learning to select actions with spiking neurons in the
  Basal Ganglia.} \\
  Frontiers in Neuroscience 6:2. DOI: 10.3389/fnins.2012.00002

T.C. Stewart, \textbf{T. Bekolay}, C. Eliasmith. (2011)
  {\sl Neural representations of compositional structures:
  representing and manipulating vector spaces with spiking neurons.} \\
  Connection Science 22, 145--153.

\section{CONFERENCE \\PROCEEDINGS}

C. M. Stille, \textbf{T. Bekolay}, B. J. Kr\"{o}ger. \\
  {\sl Neural modeling of developmental lexical disorders,}
  in Bernstein Conference.
  G\"{o}ttingen, Germany, 2017.
  DOI: 10.12751/nncn.bc2017.0126

B. J. Kr\"{o}ger, \textbf{T. Bekolay}, P.Blouw. \\
  {\sl Modeling motor planning in speech production using
    the Neural Engineering Framework,}
  in Electronic Speech Signal Processing, 15--22.
  Leipzig, Germany, 2016.

B. J. Kr\"{o}ger, \textbf{T. Bekolay}, C. Eliasmith. \\
  {\sl Modeling speech
  production using the Neural Engineering Framework,}
  in 5th IEEE Conference on Cognitive Infocommunications, 203--208.
  Vietri sul Mare, Italy, 2014.

\textbf{T. Bekolay}, C. Kolbeck, C. Eliasmith. {\sl Simultaneous
  unsupervised and supervised learning of cognitive functions
  in biologically plausible spiking neural networks,} \\ in 35th Annual
  Conference of the Cognitive Science Society, 169--174.
  Cognitive Science Society, 2013.

C. Kolbeck, \textbf{T. Bekolay}, C. Eliasmith. {\sl A biologically
  plausible spiking neuron model of fear conditioning,} in 12th International
  Conference on Cognitive Modelling, 53--58.
  Carleton University, 2013.

\section{BOOKS \& \\BOOK CHAPTERS}

B. J. Kr\"{o}ger, \textbf{T. Bekolay}. (2019) \\
  {\sl Neural Modeling of Speech Processing
  and Speech Learning: An Introduction,}
  Springer.

C. Eliasmith, \textbf{T. Bekolay}, and X. Choo. (2013) \\
  {\sl Biological cognition: Learning and memory}, in C. Eliasmith.
  ``How to build a brain: A neural architecture for biological cognition.''
  Oxford University Press. \vspace{0.42em}

{\Large \bf Conference Presentations} \\ \vspace{-8pt} \hrule

\section{ORAL \\PRESENTATIONS}

\textbf{T. Bekolay}, C. Kolbeck, C. Eliasmith.
  {\sl Simultaneous unsupervised and supervised learning of cognitive
  functions in biologically plausible spiking neural networks.} \\
  Cognitive Science 2013, in Berlin, DE.

\textbf{T. Bekolay} {\sl An efficient workflow for reproducible science.}
  SciPy 2013, in Austin, TX.

\textbf{T. Bekolay}
  {\sl A comprehensive look at representing physical quantities in Python.}
  SciPy 2013, in Austin, TX.

\textbf{T. Bekolay}
  {\sl Writing self-documenting scientific code using physical quantities.}
  PyCon Canada 2012, in Toronto, ON.

\section{POSTER \\PRESENTATIONS}

\textbf{T. Bekolay}, T. C. Stewart, X. Choo, T. DeWolf, Y. Tang,
D. Rasmussen, J. Gosmann, C. Eliasmith (2015)
  {\sl Spaun: a biologically realistic large-scale functional brain model} \\
  Ontario and Canada Research Chairs Symposium, 2015, in Toronto, ON.

\textbf{T. Bekolay}, T. C. Stewart, X. Choo, T. DeWolf, Y. Tang,
D. Rasmussen, C. Eliasmith (2013)
  {\sl Spaun: a large-scale model of the functioning brain} \\
  Cheriton Research Symposium, 2013, in Waterloo, ON. \\
  {\sl Received third prize award of \$100.}

\textbf{T. Bekolay}, B. Liu, C. Eliasmith, M. Laubach. (2012)
  {\sl A spiking neural model of strategy shifting in a simple
  reaction time task} \\
  Society for Neuroscience, 2012, in New Orleans, LA.

\textbf{T. Bekolay}, C. Eliasmith. (2011)
  {\sl A general error-modulated STDP learning rule applied to
  reinforcement learning in the basal ganglia.} \\
  Computational and Systems Neuroscience, 2011, in Salt Lake City, UT.

\textbf{T. Bekolay}, D. Metz, S. Klassen, B. Martin, P. Mahaffy. (2008) \\
  {\sl Creating interactive animations for teaching science.} \\
  NSERC Undergraduate Poster Competition, 2008, in Winnipeg, MB.
  \vspace{0.42em}

{\Large \bf Non-refereed Publications} \\ \vspace{-8pt} \hrule

\section{THESES}

\textbf{T. Bekolay.} (2016)
  {\sl Biologically inspired methods in speech recognition and synthesis:
  closing the loop.} \\
  Doctor of Philosophy thesis, University of Waterloo.

\textbf{T. Bekolay.} (2011)
  {\sl Learning in large-scale spiking neural networks.} \\
  Masters of Mathematics thesis, University of Waterloo.

\section{TECHNICAL \\REPORTS}

\textbf{T. Bekolay.} (2010) \\
  {\sl Learning nonlinear functions on vectors: examples and predictions.} \\
  Centre for Theoretical Neuroscience Technical Report CTN-TR-20101217-010.

\textbf{T. Bekolay.} (2010)
  {\sl Automating the Nengo build process.} \\
  Centre for Theoretical Neuroscience Technical Report CTN-TR-20100917-009.

\textbf{T. Bekolay.} (2010)
  {\sl Using and extending plasticity rules in Nengo.} \\
  Centre for Theoretical Neuroscience Technical Report CTN-TR-20100910-008.

\textbf{T. Bekolay.} (2010)
  {\sl A general error-based spike-timing dependent learning rule for
  the Neural Engineering Framework.} \\
  Centre for Theoretical Neuroscience Technical Report CTN-TR-20100803-006.

\section{PATENTS}

\textbf{T. Bekolay}. 2018. Methods and systems for extracting auditory features with neural networks.
  US Patent 10,026,395 B1, filed January 6, 2017, and issued July 17, 2018.

\textbf{T. Bekolay}. 2021. Methods and systems for continuous state estimation and signal classification
  with dynamic movement primitives. US Patent 10,984,309 B1, filed February 16, 2017 and issued
  April 20, 2021.

\section{SOFTWARE}

\textbf{T. Bekolay.} (2014 -- 2022)
  {\sl Lead developer of the Nengo neural simulator.}
  Released under the GPLv2 license.

\textbf{T. Bekolay.} (2010 -- Present)
  {\sl Build manager and contributor to Nengo 1.4 (an old Java version).}
  Released under Mozilla public license and GNU General Public License.

\textbf{T. Bekolay.} (2013)
  {\sl Contributions to the Matplotlib plotting library.}
  Released under a BSD license.

\textbf{T. Bekolay.} (2012 -- Present)
  {\sl Co-maintainer of the python-quantities library for physical quantities
  in Python.} Released under a BSD license.

\textbf{T. Bekolay.} (2012)
  {\sl Maintainer of the JNumeric library for matrix operations in Jython.} \\
  Released under the Python license.

\textbf{T. Bekolay.} (2011)
  {\sl Contributions to the PyBrain machine learning library.} \\
  Released under a BSD license.

\textbf{T. Bekolay,} D. Thue, V. Bulitko. (2007)
  {\sl Creator of Behaviour Tool for PaSSAGE.} \\
  Released under a BSD license.

\label{LastPage}
\end{resume}
\end{document}
