% LaTeX resume using res.cls
\documentclass[line,margin]{res}
\usepackage[T1]{fontenc}
\usepackage{charter}
\usepackage{inconsolata}
\usepackage{booktabs}

\setlength{\parskip}{0.6em}

\begin{document}

\name{\Huge{Trevor Bekolay}}
% \address used twice to have two lines of address
\address{328 Breezewood Cres, Waterloo, ON}
\address{tbekolay@gmail.com $\cdot$ (519) 573-4815}

\begin{resume}

\section{EDUCATION}

{\sl Doctor of Philosophy,} Computer Science \\
University of Waterloo, Waterloo, ON, \\
expected September 2015 \\
Member of the Computational Neuroscience Research Group \\
Cumulative GPA: 94.00

{\sl Master of Mathematics,} Computer Science \\
University of Waterloo, Waterloo, ON, \\
graduated October 2011 \\
Member of the Computational Neuroscience Research Group \\
Cumulative GPA: 91.50
  
{\sl Honours Bachelor of Computer Science, Co-op Option} \\
University of Manitoba, Winnipeg, MB, \\
graduated May 2009 \\
Minor in Asian Studies \\
Cumulative GPA: 4.28/4.5

\section{AWARDS \& \\ SCHOLARSHIPS}

\begin{tabular}{lrr}
{\sl Alexander Graham Bell CGS - Doctorate} (NSERC) &
\$105,000 & 2012/09 -- 2015/09 \\
{\sl President's Graduate Scholarship} (CS, U. of Waterloo) &
\$30,000 & 2012/09 -- 2015/09 \\
{\sl Ontario Graduate Scholarship} (Gov. of Ontario) & 
Declined & 2012/09 -- 2013/09 \\
{\sl Ontario Graduate Scholarship} (Gov. of Ontario) &
\$15,000 & 2011/09 -- 2012/09 \\
{\sl President's Graduate Scholarship} (CS, U. of Waterloo) &
\$10,000 & 2011/09 -- 2012/09 \\
{\sl OGS in Science and Technology} (Gov. of Ontario) &
\$5,000 & 2011/01 -- 2011/05 \\
{\sl Alexander Graham Bell CGS - Masters} (NSERC) &
\$17,500 & 2009/09 -- 2010/09 \\
{\sl President's Graduate Scholarship} (CS, U. of Waterloo) &
\$10,000 & 2009/09 -- 2010/09 \\
{\sl Distinguished TA Award} (CS, U. of Waterloo) &
-- & 2010/09 \\
{\sl Excellence in Teaching Assistance Award} (U. of Manitoba) &
\$300 & 2008/09 \\
{\sl Shell FUEL Scholarship} (U. of Manitoba) &
\$3,750 & 2008/09 -- 2009/04 \\
{\sl Undergraduate Student Research Award} (NSERC) &
\$4,500 & 2008/05 -- 2008/08 \\
{\sl Elizabeth Luginbuhl Memorial Award} (U. of Manitoba) &
\$2,250 & 2007/09 -- 2008/09 \\
{\sl Undergraduate Student Research Award} (NSERC) &
\$4,500 & 2007/05 -- 2007/08 \\
{\sl Japan Student Services Scholarship} (Kokugakuin, Tokyo) &
\$11,000 & 2005/08 -- 2006/08
\end{tabular}
\section{TEACHING \\EXPERIENCE}
  {\bf Sessional Instructor} ~~3 terms \vspace{4pt} \\
  David R. Cheriton School of Computer Science, University of Waterloo
  \begin{itemize}  \itemsep -2pt % reduce space between items
    \item CS230 - Introduction to Computers and Computer Systems \hfill Fall 2011
    \item CS116 - Introduction to Computer Science 2 \hfill Winter 2011
  \end{itemize} \vspace{-4pt}
  Department of Computer Science, University of Manitoba
  \begin{itemize}  \itemsep -2pt % reduce space between items
    \item COMP1010 - Introductory Computer Science 1 \hfill Fall 2008
  \end{itemize}
  
  {\bf Tutorial/Lab Instructor} ~~9 terms \vspace{4pt} \\
  David R. Cheriton School of Computer Science, University of Waterloo
  \begin{itemize}  \itemsep -2pt % reduce space between items
    \item CS230 - Intro to Comp. \& Comp. Systems \hfill Fall 2012, Fall 2011, Winter 2010
    \item CS115 - Introduction to Computer Science 1 \hfill Fall 2010
  \end{itemize} \vspace{-4pt}
  Department of Computer Science, University of Manitoba
  \begin{itemize}  \itemsep -2pt % reduce space between items
    \item COMP2160 - Programming Practices \hfill Fall 2008, Fall 2007
    \item COMP2140 - Data Structures and Algorithms \hfill Winter 2008
    \item COMP1010 - Introductory Computer Science 1 \hfill Fall 2007
    \item COMP1260 - Introductory Computer Usage 1 \hfill Fall 2006
  \end{itemize}
  
  {\bf Marker} ~~8 terms \vspace{4pt} \\
  David R. Cheriton School of Computer Science, University of Waterloo
  \begin{itemize}  \itemsep -2pt % reduce space between items
    \item CS486/686 - Introduction to Artificial Intelligence \hfill Summer 2010
    \item CS135 - Designing Functional Programs \hfill Fall 2009
  \end{itemize} \vspace{-4pt}
  Department of Computer Science, University of Manitoba
  \begin{itemize}  \itemsep -2pt % reduce space between items
    \item COMP3040 - Technical Communication in CS \hfill Winter 2009, Winter 2008
    \item COMP3620 - Professional Practice in Computer Science \hfill Winter 2009
    \item COMP2140 - Data Structures and Algorithms \hfill Winter 2008, Winter 2007
    \item COMP1010 - Introductory Computer Science 1 \hfill Fall 2006
  \end{itemize}

\section{TEACHING \\ACCREDITATION}
  {\bf Fundamentals of University Teaching} \\
  Completed in April, 2012.

  {\bf Certificate in University Teaching} \\
  In progress; expected completion in August, 2013.

\section{WORK \\EXPERIENCE}
  {\sl Staff Writer} \hfill Winter 2010 - Summer 2011 \\
  How-To Geek, LLC, Herndon, VA
  \begin{itemize}  \itemsep -2pt % reduce space between items
    \item Wrote how-to guides on various technical topics for the howtogeek.com blog.
    \item Wrote over 55 articles which have been viewed over 3.5 million times collectively.
  \end{itemize}
  
  {\sl Science and Technology Editor} \hfill 2008 - 2009 \\
  The Manitoban student newspaper, University of Manitoba
  \begin{itemize}  \itemsep -2pt % reduce space between items
    \item Prepared a weekly science and technology section.
    \item Wrote over 70 articles.
    \item Recruited several volunteer contributors and edited their submissions.
  \end{itemize}
  
  {\sl Research Assistant} \hfill Winter 2008 \\
  Centres for Research in Youth, Science Teaching and Learning (CRYSTAL)
  \begin{itemize}  \itemsep -2pt % reduce space between items
    \item Created several interactive animations for teaching science concepts to students.
    \item Collaborated with a CRYSTAL group in Edmonton, AB.
  \end{itemize}
  
  {\sl Research Assistant} \hfill Winter 2007 \\
  Centres for Research in Youth, Science Teaching and Learning (CRYSTAL)
  \begin{itemize}  \itemsep -2pt % reduce space between items
    \item Assisted with the PaSSAGE (Player-Specific Stories via Automatically Generated Events) project.
    \item Created a GUI tool in Python to aid in the development of PaSSAGE.
  \end{itemize}

\section{OTHER \\ ACTIVITIES}
  Co-founder and co-chair of the Cheriton Review committee,
  University of Waterloo. 2011-2012.

  TA Committee member of the CS-GSA, University of Waterloo. \\
  2010-2012.

  University of Ottawa Summer School in Computational Neuroscience. \\
  June 13-25, 2010.

\clearpage

\section{REFEREED \\PUBLICATIONS}
  Eliasmith, C., Stewart, T.C., Choo, X., \textbf{Bekolay, T.},
    DeWolf, T., Tang, C., Rasmussen, D. (2012)
    {\sl A large-scale model of the functioning brain.} \\
    Science 338:6111, 1202-1205. DOI: 10.1126/science.1225266

  Stewart, T.C., \textbf{Bekolay, T.}, Eliasmith, C. (2012)
    {\sl Learning to select actions with spiking neurons in the Basal Ganglia.} \\
    Frontiers in Neuroscience 6:2. DOI: 10.3389/fnins.2012.00002

  Stewart, T.C., \textbf{Bekolay, T.}, Eliasmith, C. (2011)
    {\sl Neural representations of compositional structures: representing and manipulating vector spaces with spiking neurons.} \\
    Connection Science 22, 145-153.

\section{IN PRESS}
  Eliasmith, C., \textbf{Bekolay, T.}, and Choo, X. (2011) \\
   {\sl Biological cognition: Learning and memory}, in Eliasmith, C. ``How to build a brain: A neural architecture for biological cognition.'' Oxford University Press.

\section{POSTERS \\PRESENTED}
  \textbf{Bekolay, T.}, Liu, B., Eliasmith, C., Laubach, M. (2012)
    {\sl A spiking neural model of strategy shifting in a simple reaction time task} \\
    Society for Neuroscience, 2012, in New Orleans, LA.

  \textbf{Bekolay, T.}, Eliasmith, C. (2011)
    {\sl A general error-modulated STDP learning rule applied to reinforcement learning in the basal ganglia.} \\
    Computational and Systems Neuroscience, 2011, in Salt Lake City, UT.
  
  \textbf{Bekolay, T.}, Metz, D., Klassen, S., Martin, B., Mahaffy, P. (2008) \\
    {\sl Creating interactive animations for teaching science.} \\
    NSERC Undergraduate Poster Competition, 2008, in Winnipeg, MB.

\section{NON-REFEREED \\PUBLICATIONS}
  \textbf{Bekolay, T.} (2011)
    {\sl Learning in large-scale spiking neural networks.} \\
    Masters of Mathematics thesis, University of Waterloo.

  \textbf{Bekolay, T.} (2010) \\
    {\sl Learning nonlinear functions on vectors: examples and predictions.} \\
      Centre for Theoretical Neuroscience Technical Report CTN-TR-20101217-010.

  \textbf{Bekolay, T.} (2010)
    {\sl Automating the Nengo build process.} \\
    Centre for Theoretical Neuroscience Technical Report CTN-TR-20100917-009.

  \textbf{Bekolay, T.} (2010)
    {\sl Using and extending plasticity rules in Nengo.} \\
    Centre for Theoretical Neuroscience Technical Report CTN-TR-20100910-008.

  \textbf{Bekolay, T.} (2010)
    {\sl A general error-based spike-timing dependent learning rule for the Neural Engineering Framework.} \\
    Centre for Theoretical Neuroscience Technical Report CTN-TR-20100803-006.

\section{SOFTWARE}
  \textbf{Bekolay, T.} (2010 - 2012)
    {\sl Build manager and contributor to the Nengo neural simulator.}
    Released under a Mozilla public license.

  \textbf{Bekolay, T.} (2012)
    {\sl Co-maintainer of the python-quantities library for physical quantities in Python.} Released under a BSD license.

  \textbf{Bekolay, T.} (2012)
    {\sl Maintainer of the JNumeric library for matrix operations in Jython.} \\
    Released under the Python license.

  \textbf{Bekolay, T.} (2011)
    {\sl Contributions to the PyBrain machine learning library.} \\
    Released under a BSD license.

  \textbf{Bekolay, T.} Thue, D., Bulitko, V.. (2007)
    {\sl Creator of Behaviour Tool for PaSSAGE.} \\
    Released under a BSD license.

\end{resume}
\end{document}

